\documentclass[12pt]{article}
\usepackage{amsmath}
\usepackage{amsfonts}
\usepackage{vmargin}
\usepackage[utf8]{inputenc}

\setmargins{2.5cm}
{0.1cm}
{16.5cm}
{23.42cm}
{10pt}
{1cm}
{0pt}
{2cm} 

\begin{document}
\title{Ayudant\'ia N.7}
\date{21 de Octubre 2022}
\author{Daniel S\'anchez}
\maketitle

\begin{enumerate}
    \item Determine cu\'al de los siguientes subconjuntos de sus correspondientes son subespacios vectoriales:
          \begin{enumerate}
              \item $W_1=\left\{\begin{bmatrix}
                            a & 0 \\
                            0 & b
                        \end{bmatrix}|\mbox{ }a,b\in \mathbb{R}\right\}$
              \item $W_2=\left\{\begin{bmatrix}
                            a & 1 \\
                            1 & b
                        \end{bmatrix}|\mbox{ }a,b\in \mathbb{R}\right\}$
              \item $W_3=\left\{
                        (x,y,z)\in \mathbb{R}^3|\mbox{ }y=z-3x\right\}$
          \end{enumerate}

    \item Considere el conjunto $C=\left\{A\in M_2(\mathbb{R})/BA=-AB\right\}$, donde
          $B=\begin{bmatrix}
                  3  & 0 \\
                  -1 & 4
              \end{bmatrix}$. \\Determine si $C$ es un subespacio vectorial de $M_2(\mathbb{R})$

    \item Sean $v=(1,-3,2)$ y $w=(2,-1,1)$
          \begin{enumerate}
              \item Exprese $(1,7,-4)$ como combinaci\'on lineal de $v \mbox{ y } w$.
              \item ¿Para qu\'e valores de $k \in \mathbb{R}$ el vector $(1,k,5)$ es combinaci\'on
                    lineal de $v \mbox{ y } w$.
              \item Determine las condiciones sobre $a,b,c \in \mathbb{R}$ para que el vector $(a,b,c)$ sea
                    combinaci\'on lineal de $v \mbox{ y } w$.
          \end{enumerate}

    \item Determine si los siguientes conjuntos de vectores son linealmente independientes o
          dependientes. En caso de ser LD, exprese uno de los vectores como combinaci\'on lineal de los dem\'as:
          \begin{enumerate}
              \item $\left\{x^3-x^2-x+2,3x^3-x^2+5,x^3+x^2+2x+1\right\}$
              \item $\left\{(1,0,1),(1,1,0),(0,1,-1),(1,0,2)\right\}$
              \item $\left\{
                        \begin{bmatrix}
                            1 & -1 \\
                            1 & 0
                        \end{bmatrix},
                        \begin{bmatrix}
                            0  & 1 \\
                            -1 & 1
                        \end{bmatrix},
                        \begin{bmatrix}
                            0  & 0  \\
                            -1 & -1
                        \end{bmatrix},
                        \begin{bmatrix}
                            2 & 0 \\
                            0 & 2
                        \end{bmatrix}\right\}$
          \end{enumerate}
\end{enumerate}
\end{document}