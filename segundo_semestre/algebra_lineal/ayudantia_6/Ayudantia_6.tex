\documentclass[12pt]{article}
\usepackage{amsmath}
\usepackage{amsfonts}
\usepackage{vmargin}
\usepackage[utf8]{inputenc}
\usepackage{graphicx}

\graphicspath{}

% \displaystyle is sooo important
\newcommand{\D}{\displaystyle}

\setmargins{2.5cm}
{0.1cm}
{16.5cm}
{23.42cm}
{10pt}
{1cm}
{0pt}
{2cm} 

\begin{document}
\title{
    \begin{figure}[ht]
        \centering
        \includegraphics[width = 0.4\textwidth, ]{../../logo-uai.jpg}\\
    \end{figure}
    Ayudant\'ia \'Algebra Lineal N.6}
\date{7 de Octubre 2022}
\author{Daniel S\'anchez}
\maketitle

\begin{enumerate}

    \item Considere la matriz $A=\begin{bmatrix}
                  k & k & 1 \\
                  1 & 1 & 1 \\
                  2 & 1 & 2
              \end{bmatrix}$
          \begin{enumerate}
              \item Determine para qu\'e valores de $k \in \mathbb{R}$, la matriz A es invertible.
              \item Considerando $k=2$, calcule, si se puede, $A^{-1}$, a trav\'es del Cof$(A)$ y la Adj$(A)$.
          \end{enumerate}

    \item Sea
          $$A=\begin{bmatrix}
                  5 & -5 & 1 \\
                  0 & 2  & 3 \\
                  1 & 1  & 1
              \end{bmatrix}$$
          \begin{enumerate}
              \item Determine $A^{-1}$.
              \item Escriba $A^{-1}$ como producto de matrices elementales.
              \item Escriba $A$ como producto de matrices elementales.
          \end{enumerate}

    \item Dado el SEL:
          $$\left\{\begin{array}{rcc}
                  x+z+w   & = & 5 \\
                  x-z+w   & = & 1 \\
                  x+y+z+w & = & 3 \\
                  2x+2z   & = & 2
              \end{array}\right.$$
          \begin{enumerate}
              \item Determine su soluci\'on particular por medio de factorizaci\'on $LU$.
              \item Determine su soluci\'on homog\'enea.
              \item Determine su soluci\'on general.
          \end{enumerate}

\end{enumerate}

\pagebreak
\begin{center}
    \textbf{\underline{Tips}}
\end{center}
\begin{enumerate}
    \item Elementales:
          \subitem
          $\begin{array}{lrcl}
                  \mbox{Cambio de filas:}              & (E_{ij})^{-1}    & = & E_{ij}           \\
                  \mbox{Multiplicaci\'on por escalar:} & (E_i(r))^{-1}    & = & E_i(\frac{1}{r}) \\
                  \mbox{Operaci\'on fila:}             & (E_{ij}(k))^{-1} & = & E_{ij}(-k)
              \end{array}$
    \item Estructura de matriz normal e inversa con matrices elementales:
          \subitem
          $\begin{array}{lccl}
                  \mbox{Matriz normal:}  & (E_{1})^{-1}\cdot(E_{2})^{-1}\cdot...\cdot(E_{(n-1)})^{-1}\cdot(E_{(n)})^{-1} & = & A      \\
                  \mbox{Matriz inversa:} & (E_{(n)}\cdot E_{(n-1)}\cdot...\cdot E_{2}\cdot E_{1})                        & = & A^{-1}
              \end{array}$
    \item Aplicaci\'on de la inversa:
          $$(ABC)^{-1}=C^{-1}(AB)^{-1}=C^{-1}B^{-1}A^{-1}$$
          $$(ABCD)^{-1}=(CD)^{-1}(AB)^{-1}=D^{-1}C^{-1}B^{-1}A^{-1}$$
    \item Cantidad de par\'ametros:
          \subitem $n$: N\'umero de inc\'ognitas
          \subitem $r$: Rango de la matriz
          \subitem $p$: Par\'ametros
          $$(n-r)=p$$
    \item Soluci\'on general:
          \subitem $X_G$: Soluci\'on general
          \subitem $X_P$: Soluci\'on particular
          \subitem $X_H$: Soluci\'on homog\'enea
          $$X_G=X_P+X_H$$
    \item Determinante y matrices elementales:
          \subitem $\begin{array}[]{crcc}
                  \mbox{Propiedades:} & |E_{ij}(k)| & = & 1  \\
                                      & |E_{i}(r)|  & = & r  \\
                                      & |E_{ij}|    & = & -1 \\
                                      &             &   &
              \end{array}$
          \subitem $\begin{array}[]{lcl}
                  |A| & = & |(E_{1})^{-1}\cdot(E_{2})^{-1}\cdot...\cdot(E_{(n-1)})^{-1}\cdot(E_{(n)})^{-1}|       \\
                      & = & |(E_{1})^{-1}|\cdot|(E_{2})^{-1}|\cdot...\cdot|(E_{(n-1)})^{-1}|\cdot|(E_{(n)})^{-1}|
              \end{array}$
\end{enumerate}
\end{document}