\documentclass[12pt]{article}
\usepackage{amsmath}
\usepackage{amsfonts}
\usepackage{vmargin}
\usepackage[utf8]{inputenc}
\usepackage{graphicx}

\graphicspath{}

\setmargins{2.5cm}
{0.1cm}
{16.5cm}
{23.42cm}
{10pt}
{1cm}
{0pt}
{2cm} 

\begin{document}
\title{
    \begin{figure}[ht]
        \centering
        \includegraphics[width = 0.4\textwidth, ]{../../logo-uai.jpg}\\
    \end{figure}
    Ayudant\'ia \'Algebra Lineal N.1}
\date{12 de Agosto 2022}
\author{Daniel S\'anchez}
\maketitle

\begin{enumerate}
    \item Factorice el polinomio $p(x) = x^4-2x^3+x^2+2x-2$ en $\mathbb{C}\left[x\right]$, si se sabe que
          $x-1+i$ es uno de sus factores y determine sus ra\'ices.
    \item Sea $p(x) = x^5-x^3+2x^2-6x+4$
          \begin{enumerate}
              \item Factorice $p(x)$ en factores irreducibles de $\mathbb{R}\left[x\right]$.
              \item Determine todas las ra\'ices del polinomio $p(x)$ en $\mathbb{C}\left[x\right]$.
          \end{enumerate}
          
    \item Sea la matriz $A=(a_{ij})_{3x3}$ tal que:
          $a_{ij}=
              \left\{ \begin{array}{ccr}
                  i-j & \mbox{si} & i > j    \\
                  ij  & \mbox{si} & i \leq j
              \end{array}
              \right.$
          \begin{enumerate}
              \item Determine $A^t$
              \item Calcule la traza de $A$.
          \end{enumerate}
    \item Dos familias planean ir a comprar manzanas, naranjas y peras a dos fruter\'ias distintas.
          La siguiente tabla muestra las cantidades de cada fruta que requiere cada familia:
          \begin{table}[h]
              \centering
              \begin{tabular}{|l|c|c|}
                  \hline
                           & Familia 1 & Familia 2 \\ \hline
                  Manzanas & $6$       & $4$       \\ \hline
                  Naranjas & $6$       & $4$       \\ \hline
                  Peras    & $6$       & $4$       \\ \hline
              \end{tabular}
          \end{table}\\
          Los precios por unidad de cada tipo de frutas, en cada fruter\'ia, est\'an dados por:
          \begin{table}[h]
              \centering
              \begin{tabular}{|l|c|c|}
                  \hline
                  \textit{} & Fruter\'ia 1 & Fruter\'ia 2 \\ \hline
                  Manzanas  & $\$80$       & $\$75$       \\ \hline
                  Naranjas  & $\$95$       & $\$85$       \\ \hline
                  Peras     & $\$75$       & $\$80$       \\ \hline
              \end{tabular}
          \end{table}
          \\
          Utilice \'algebra matricial para calcular el costo total por la compra de estas 
          frutas en cada fruter\'ia por familia.
\end{enumerate}
\end{document}