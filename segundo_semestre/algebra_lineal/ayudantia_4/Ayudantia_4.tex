\documentclass[12pt]{article}
\usepackage{amsmath}
\usepackage{amsfonts}
\usepackage{vmargin}
\usepackage[utf8]{inputenc}
\usepackage{graphicx}

\graphicspath{}

% \displaystyle is sooo important
\newcommand{\D}{\displaystyle}

\setmargins{2.5cm}
{0.1cm}
{16.5cm}
{23.42cm}
{10pt}
{1cm}
{0pt}
{2cm} 

\begin{document}
\title{
    \begin{figure}[ht]
        \centering
        \includegraphics[width = 0.4\textwidth, ]{../../logo-uai.jpg}\\
    \end{figure}
    Ayudant\'ia \'Algebra Lineal N.4}
\date{9 de Septiembre 2022}
\author{Daniel S\'anchez}
\maketitle

\begin{enumerate}

    \item Una acuicultura entrega tres tipos de alimento a un embalse que alberga
          a tres especies de peces.\\
          Cada pez de la especie $alpha$ consume cada semana 1 unidad del alimento $A$, 1 unidad
          del alimento $B$ y 2 unidades del alimento $C$. Cada pez de la especie $beta$ consume semanalmente
          3 unidades del alimento $A$, 4 del $B$ y 5 del $C$. Para un pez de la especie $gamma$ el consumo
          semanal es de 2 unidades del alimento $A$, 1 unidad del $B$ y 5 unidades del $C$.\\
          Cada semana se vierten en el embalse 25.000 unidades del alimento $A$, 20.000 unidades del alimento
          $B$ y 55.000 unidades del alimento $C$. Si suponemos que los peces consumen todo el alimento ¿Cu\'antos peces
          de cada especie pueden coexistir en el embalse?
          
    \item Considere la matriz:
          $$A=\begin{bmatrix}
                  -1 & 2 & 2  & 1 & 0  \\
                  1  & 4 & -3 & 2 & 2  \\
                  0  & 2 & 1  & 1 & -1 \\
                  2  & 2 & 1  & 1 & 1
              \end{bmatrix}$$
          Una matriz que corresponde a la matriz de coeficientes de un sistema homog\'eneo:
          \begin{enumerate}
              \item Calcule el rango de la matriz A
              \item Encuentre la soluci\'on homog\'enea para A.
          \end{enumerate}
          
    \item Hallar para qu\'e valores de 'a' el siguiente sistema:
          $$\begin{array}{rcc}
                  x+y-z    & = & 1 \\
                  x-y+z    & = & 7 \\
                  2x+ay-4z & = & a
              \end{array}$$
          \begin{enumerate}
              \item No tiene soluci\'on.
              \item Tiene infinitas soluciones.
              \item Tiene soluci\'on \'unica.
          \end{enumerate}
          
\end{enumerate}
\end{document}