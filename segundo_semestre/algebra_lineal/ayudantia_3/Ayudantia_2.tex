\documentclass[12pt]{article}
\usepackage{amsmath}
\usepackage{amsfonts}
\usepackage{vmargin}
\usepackage[utf8]{inputenc}
\usepackage{graphicx}

\graphicspath{}

\setmargins{2.5cm}
{0.1cm}
{16.5cm}
{23.42cm}
{10pt}
{1cm}
{0pt}
{2cm} 

\begin{document}
\title{
    \begin{figure}[ht]
        \centering
        \includegraphics[width = 0.4\textwidth, ]{../../logo-uai.jpg}\\
    \end{figure}
    Ayudant\'ia \'Algebra Lineal N.3}
\date{26 de Agosto 2022}
\author{Daniel S\'anchez}
\maketitle

\begin{enumerate}
    \item Sean $A= \begin{bmatrix}
                  -1 & 0 & 2 \\
                  2  & 1 & 0
              \end{bmatrix}$ y $B=\begin{bmatrix}
                  -1 & 0 \\
                  1  & 2
              \end{bmatrix}$
          \begin{enumerate}
              \item Calcule $B\cdot A$
              \item Encuentre la matriz $X \in M_{2x2}(\mathbb{R})$ tal que:
                    $$A\cdot A^t+B\cdot X=(3I_2-2B)^t$$
          \end{enumerate}
          % AYUDANTIA 3
          % \item Una empresa produce tres tipos de productos: $x_1, x_2 \mbox{ y }x_3$. La utilidad
          %       se obtiene mediante la siguiente relaci\'on:
          %       $$G(x_1,x_2,x_3)=20x_1-5x_2+10x_3$$
          %       y las cantidades a producir deben cumplir con las siguientes restricciones:
          %       $$\begin{array}{rcl}
          %               x_1 - x_2 + 8x_3    & = & 50 \\
          %               2x_1 + 3x_2 + 11x_3 & = & 60 \\
          %               x_2 + 5x_3          & = & 30
          %           \end{array}$$
          %       \begin{enumerate}
          %           \item Resuelva el sistema, considerando que $x_1,x_2,x_3 \in \mathbb{{Z^+}_0}$
          %           \item Considere la soluci\'on en a) para obtener la utilidad.
          %       \end{enumerate}
          
\end{enumerate}
\end{document}