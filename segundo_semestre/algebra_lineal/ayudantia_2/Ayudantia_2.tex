\documentclass[12pt]{article}
\usepackage{amsmath}
\usepackage{amsfonts}
\usepackage{vmargin}
\usepackage[utf8]{inputenc}
\usepackage{graphicx}

\graphicspath{}

\setmargins{2.5cm}
{0.1cm}
{16.5cm}
{23.42cm}
{10pt}
{1cm}
{0pt}
{2cm} 

\begin{document}
\title{
    \begin{figure}[ht]
        \centering
        \includegraphics[width = 0.4\textwidth, ]{../../logo-uai.jpg}\\
    \end{figure}
    Ayudant\'ia \'Algebra Lineal N.2}
\date{19 de Agosto 2022}
\author{Daniel S\'anchez}
\maketitle

\begin{enumerate}
    \item Considere las matrices
          $A=
              \begin{bmatrix}
                  1 & -1 \\
                  0 & 1
              \end{bmatrix}$
          y
          $B=
              \begin{bmatrix}
                  0  & -1 \\
                  -1 & 0
              \end{bmatrix}$.
          \begin{enumerate}
              \item Calcule $B^2$
              \item Determine $Y \in M_2(\mathbb{R})$
          \end{enumerate}
          $$ (3Y-2AB)^t=2A-3(AB)^t$$
          
          
    \item Resolver las siguientes ecuaciones matriciales en $X$:
          \begin{enumerate}
              \item $ABXA^{-1}B=I+B$, donde $A=
                        \begin{bmatrix}
                            -1 & 0  \\
                            0  & -1
                        \end{bmatrix}$
                    y
                    $B=
                        \begin{bmatrix}
                            2 & 1  \\
                            0 & -1
                        \end{bmatrix}$
                    
              \item Si $A, B, C$ y $X$ son matrices cuadradas de orden n tales que:
                    \begin{itemize}
                        \item $A$ es sim\'etrica e invertible
                        \item $B$ es antisim\'etrica e invertible
                        \item $X$ es sim\'etrica
                    \end{itemize}
                    $$(AB)^{-1}(XA+B)^tA=CA$$
          \end{enumerate}
    \item Considere la ecuaci\'on matricial:
          $$B^t[B^{-1}+X^t]=[I+A]A^t$$
          con $A$ y $B$ matrices cuadradas de orden $n$
          y donde se cumple que $AA^t=I_n\mbox{ y }B$ es
          sim\'etrica e invertible. Resuelva para $X$ la ecuaci\'on
          matricial dada y luego demuestre que:
          $$X^{-1}=BA^t$$
          
          
          % AYUDANTIA 3
          % \item Una empresa produce tres tipos de productos: $x_1, x_2 \mbox{ y }x_3$. La utilidad
          %       se obtiene mediante la siguiente relaci\'on:
          %       $$G(x_1,x_2,x_3)=20x_1-5x_2+10x_3$$
          %       y las cantidades a producir deben cumplir con las siguientes restricciones:
          %       $$\begin{array}{rcl}
          %               x_1 - x_2 + 8x_3    & = & 50 \\
          %               2x_1 + 3x_2 + 11x_3 & = & 60 \\
          %               x_2 + 5x_3          & = & 30
          %           \end{array}$$
          %       \begin{enumerate}
          %           \item Resuelva el sistema, considerando que $x_1,x_2,x_3 \in \mathbb{{Z^+}_0}$
          %           \item Considere la soluci\'on en a) para obtener la utilidad.
          %       \end{enumerate}
          
          
          
    \item Demuestre que $\forall a \in \mathbb{R}$,
          la matriz $A=\begin{bmatrix}
                  a     & {1+a} \\
                  {1-a} & -a
              \end{bmatrix}$ es invertible y $A^{-1}=A$
          
    \item \textbf{Python}: Hay dos tiendas en el Mall UAI, La tienda Azul y la tienda
          Roja. Cada una de las tiendas tienen juguetes, libros, l\'apices y cuadernos.
          La tienda azul le asign\'o el valor de \$800 a los juguetes, \$1090 a los libros,
          \$250 a los l\'apices y \$590 a los cuadernos. Por otro lado la tienda roja le asigno,
          en orden correspondiente, los valores de \$750, \$1050, \$350 y \$500.
          La tienda Azul y la tienda Roja casualmente vendieron las mismas cantidades de cada producto,
          250 juguetes, 500 libros, 420 l\'apices y 820 cuadernos.
          
          
          Defina dos matrices cuyo producto sea el ingreso total correspondiente
          para cada tienda.
          
    \item \textbf{Python}:
          Considere las siguientes matrices A y B. 
          $$ A=\begin{bmatrix}
                  2  & 2  & 3            \\
                  2  & 1  & 3            \\
                  -1 & -1 & 0           
              \end{bmatrix}
              B=\begin{bmatrix}
                  7 & 2 & 0              \\
                  5 & 1 & -3             \\
                  2 & 4 & 1             
              \end{bmatrix}  $$
          
          Escriba un código que calcule $B^{2}+A^{T}$
\end{enumerate}
\end{document}