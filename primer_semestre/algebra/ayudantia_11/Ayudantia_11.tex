\documentclass[12pt]{article}
\usepackage{amsmath}
\usepackage{amsfonts}
\usepackage{amssymb}
\usepackage{vmargin}
\usepackage[utf8]{inputenc}

% \displaystyle is sooo important
\newcommand{\D}{\displaystyle}

\begin{document}
\setmargins{2.5cm}
{0.1cm}
{16.5cm}
{23.42cm}
{10pt}
{1cm}
{0pt}
{2cm}

\title{Ayudant\'ia \'Algebra N.11}
\date{3 de Junio 2022}
\author{Daniel S\'anchez}
\maketitle

\begin{enumerate}
      \item Dado los puntos $A(-3,4,-1),B(1,-1,1)$ y $C(-1,2,3)$ de $\mathbb{R}^3$. Si $\vec{u}=\Vec{AB}$ y $\vec{v}=\vec{CA}$ determine el valor de $x,y \in \mathbb{R}$, si existe, de modo que: $$3(\vec{u}-2\vec{v})+4\vec{v}=(2x-y,x+y,14)$$
      \item Considere los vectores $\vec{a}=(3,-1,0)$ y $\vec{b}=(5,-1,2)$ de $\mathbb{R}^3$. Determine si los vectores $\vec{a}$ y $\vec{b}$ son paralelos. Justifique.
      \item Sean $\vec{u}=(3,-2,1),\vec{v}=(1,2,-3)$ y $\vec{w}= -\hat{i}+\hat{j}$, vectores de $\mathbb{R}^3$.
            \begin{enumerate}
                  \item Calcule el ángulo entre los vectores $\vec{u}$ y $\vec{v}$.
                  \item Determine $\alpha \in \mathbb{R}$ de modo que $\alpha \cdot u + v$ sea perpendicular a $w$.
            \end{enumerate}
      \item ¿Las rectas $l_1 = \dfrac{x-1}{2}=\dfrac{y-3}{2}=z-1$ y $l_2:(x,y,z)=(3,4,2)+t(2,1,1), t \in \mathbb{R}$, son secantes? Si lo son, encuentre el punto intersección.
            
\end{enumerate}
\pagebreak
\textbf{Propiedades}

\begin{itemize}
      \item $\vec{a} \cdot \vec{a} = \Vert \vec{a} \Vert ^2$
      \item $\vec{a} \cdot \vec{b} = \vec{b} \cdot \vec{a}$
      \item $\vec{a} \cdot (\mbox{ }\vec{b} + \vec{c}\mbox{ }) = \vec{a} \cdot \vec{b} + \vec{a} \cdot \vec{c}$
      \item $\alpha \mbox{ }(\vec{a}\cdot \vec{b}) = (\alpha \mbox{ }\vec{a})\cdot \vec{b} = (\alpha \mbox{ }\vec{b})\cdot \vec{a}$
      \item $\vec{a} \cdot \vec{b} = \Vert \vec{a} \Vert \cdot \Vert \vec{b} \Vert \cos({\theta})$
      \item $\vec{a}\cdot \vec{b} = \frac{1}{4}\Vert \vec{a} + \vec{b}\Vert ^2 - \frac{1}{4}\Vert \vec{a} - \vec{b}\Vert ^2$
      \item $\vec{a}\cdot \vec{b} = \frac{1}{2}(\Vert \vec{a} \Vert ^2 + \Vert \vec{b} \Vert ^2 - \Vert \vec{a} - \vec{b}\Vert ^2)$
\end{itemize}

\end{document}