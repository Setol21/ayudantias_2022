\documentclass[12pt]{article}
\usepackage{amsmath}
\usepackage{amsfonts}
\usepackage{amssymb}
\usepackage{vmargin}
\usepackage[utf8]{inputenc}

% \displaystyle is sooo important
\newcommand{\D}{\displaystyle}

\begin{document}
\setmargins{2.5cm}
{0.1cm}
{16.5cm}
{23.42cm}
{10pt}
{1cm}
{0pt}
{2cm}

\title{Ayudant\'ia \'Algebra N.9}
\date{13 de Mayo 2022}
\author{Daniel S\'anchez}
\maketitle

\begin{enumerate}
      \item Resuelva las siguientes ecuaciones trigonom\'etricas:
            \begin{enumerate}
                  \item $\arccos(x) - \arcsin(x) = \arcsin(\sqrt{3}x)$
                  \item $2(\cos^2(x) - \sin^2(x)) = -1$
                  \item $\arcsin(x) - \arccos(x) = \arcsin(10x-9)$
            \end{enumerate}
      \item Sea $ABC$ un tri\'angulo cualquiera. Pruebe que:
            $$2(bc\cos(\alpha)+ac\cos(\beta)+ab\cos(\gamma)) = a^2+b^2+c^2$$
      \item Un niño est\'a haciendo volar dos cometas simult\'aneamente. Una de ellas tiene 38
            metros de cuerta y la otra 42 metros. Si el \'angulo entre las dos cuerdas es de 30°,
            estime la distancia entre los dos cometas.
            % \item Un avi\'on vuela en l\'inea recta a una altura de 1000 metros. A las 13 horas se encuentra
            %       en $A$ y asciende bruscamente desvi\'andose 30° con la horizontal, manteniendo el movimiento
            %       rectil\'ineo con rapidez constante. Despu\'es de 10 segundos el avi\'on se encuentra en $B$. 
            %       Si desde una torre de observaci\'on en la tierra, se tienen \'angulos de elevaci\'on de 30° y 45° 
            %       para los puntos $A$ y $B$ respectivamente, indicar cu\'al es la rapidez del avi\'on suponiendo que 
            %       las visuales de los puntos $A$ y $B$ y la trayectoria del avi\'on est\'an en un plano.
\end{enumerate}
\end{document}