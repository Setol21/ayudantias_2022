\documentclass[12pt]{article}
\usepackage{amsmath}
\usepackage{amsfonts}
\usepackage{amssymb}
\usepackage{vmargin}
\usepackage[utf8]{inputenc}

\begin{document}
\setmargins{2.5cm}
{0.1cm}
{16.5cm}
{23.42cm}
{10pt}
{1cm}
{0pt}
{2cm}

\title{Ayudant\'ia \'Algebra N.2}
\date{18 de Marzo 2022}
\author{Daniel S\'anchez}
\maketitle

\begin{enumerate}
    \item Determine si las siguientes proposiciones corresponden a una tautolog\'ia, contradicci\'on o contingencia:
          \begin{enumerate}
              \item $[(p \Rightarrow q) \land (p \land \neg{q} \land r)] \Rightarrow (\neg{p} \lor q)$
              \item $[\{(p \lor q) \land \neg{p}\}\Rightarrow q] \Leftrightarrow q$
          \end{enumerate}
    \item Simplificar, aplicando propiedades:
          $$[A \cap B^{c} \cap (A-B^{c})]^{c} \cup A^{c}$$
    \item Dados los conjuntos $A, B \mbox{ y } C$, simplificar al m\'aximo la siguiente expresi\'on:
          $$[A \cap (A^{c} \cup B)] \cup [B \cap (B \cup C)] \cup B$$
    \item Considere el conjunto referencial $E=\mathbb{N}$ y las funciones proposicionales:
          $$p(x):x\mbox{ es par, } q(x):x\geq 5$$
          Decida si las proposiciones son ambas verdaderas:
          $$\forall x \in E: p(x) \Rightarrow \neg q(x)\mbox{ ; } \exists x \in E: p(x)$$
    \item Sean $A=\{-1,2,3,4,1\} \mbox{ y } B=\{1,\pi\}$. Determine el valor de verdad de la afirmaci\'on. Justifique.
          $$\exists x \in A, \forall y \in B:x+y<y$$
    \item Determine el valor de verdad de las siguientes proposiciones:
          \begin{enumerate}
              \item $(\exists x \in \{2,3,4\})(\forall y \in \{-1,0,1\})(y\geq0 \Rightarrow x+y\geq 3)$
              \item $(\forall x \in \{1,-2,\frac{1}{2}\})(\exists y \in \{-1,1,2,3\})(xy > \frac{3}{2})$
          \end{enumerate}
\end{enumerate}
\pagebreak
\LARGE
\underline{\textbf{\title{Tips}}}:
\normalsize
Sean $A, B \mbox{ y } C$ conjuntos:
\begin{table}[h]
    \centering
    \begin{tabular}{|l|l|}
        \hline
        Identidad       & \begin{tabular}[c]{@{}l@{}}$A \cap \mathcal{U} = A$ \\
            $A \cap \varnothing = \varnothing$   \\
            $A \cup \mathcal{U} =\mathcal{U}$    \\
            $A \cup \varnothing = A$\end{tabular}  \\ \hline
        Idempotencia    & \begin{tabular}[c]{@{}l@{}}$A \cap A=A$ \\
            $A \cup A=A$\end{tabular}  \\ \hline
        Involuci\'on    & \begin{tabular}[c]{@{}l@{}}$(A^{c})^{c}=A$\end{tabular}  \\ \hline
        Complemento     & \begin{tabular}[c]{@{}l@{}}$A \cap A^{c} = \varnothing$ \\
            $A \cup A^{c}= \mathcal{U}$\end{tabular}  \\ \hline
        Conmutatividad  & \begin{tabular}[c]{@{}l@{}}$A \cap B= B \cap A$ \\
            $A \cup B=B \cup A$\end{tabular}  \\ \hline
        Asociatividad   & \begin{tabular}[c]{@{}l@{}}$A \cap (B \cap C)= (A \cap B) \cap C$ \\
            $A \cup (B \cup C)=(A \cup B) \cup C$\end{tabular}  \\ \hline
        Distributividad & \begin{tabular}[c]{@{}l@{}}$A \cap (B\cup C)=(A\cap B)\cup (A\cap C)$ \\
            $A \cup (B\cap C)=(A\cup B)\cap (A\cup C)$\end{tabular}  \\ \hline
        Leyes de Morgan & \begin{tabular}[c]{@{}l@{}}$(A\cap B)^{c}=A^{c}\cup B^{c}$ \\
            $(A\cup B)^{c}=A^{c}\cap B^{c}$\end{tabular} \\ \hline
        Absorci\'on     & \begin{tabular}[c]{@{}l@{}}$A\cap (A\cup B)=A$ \\
            $A \cup (A\cap B)=A$\end{tabular} \\ \hline
        Resta           & \begin{tabular}[c]{@{}l@{}}$A-B=A \cap B^{c}$ \\
        \end{tabular} \\ \hline
    \end{tabular}
\end{table}
\begin{enumerate}
    \item Recuerda que el s\'imbolo $\subseteq$, significa que el conjunto de la izquierda es un subconjunto del de la derecha.
\end{enumerate}
\end{document}