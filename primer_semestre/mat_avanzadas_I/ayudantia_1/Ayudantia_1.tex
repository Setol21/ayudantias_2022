\documentclass[12pt]{article}
\usepackage{amsmath}
\usepackage{amsfonts}
\usepackage{vmargin}
\usepackage[utf8]{inputenc}

\begin{document}
\setmargins{2.5cm}
{0.1cm}
{16.5cm}
{23.42cm}
{10pt}
{1cm}
{0pt}
{2cm}

\title{Ayudant\'ia Matem\'aticas Avanzadas I N.1}
\date{10 de Marzo 2022}
\author{Daniel S\'anchez}
\maketitle

\begin{enumerate}
    \item Determine si las siguientes frases son una proposici\'on, de ser
          as\'i identifique su valor de verdad:
          \begin{enumerate}
              \item Si $6<3$ entonces $30>10$.
              \item Si el cuadrado de $7$ es un n\'umero par entonces $7$ es par.
              \item $5$ es distinto de $6$ y $24$ es mayor que $36$.
              \item ¿Qu\'e d\'ia es hoy?
          \end{enumerate}
    \item Determine si las siguientes expresiones son tautolog\'ias, contradicci\'on
          o contingencia:
          \begin{enumerate}
              \item $(p \land (p \Rightarrow q)) \Rightarrow q$
              \item $[(p \Rightarrow q) \land p \land \lnot q] \Rightarrow (\lnot p \lor q)$
              \item $[(p \Rightarrow q) \land (q \Rightarrow r)]\Rightarrow (p \Rightarrow r)$
              \item $\overline{(p \lor q)} \Leftrightarrow (\overline{p} \land \overline{q})$
                    % Agregar más de estos ejercicios.
          \end{enumerate}
\end{enumerate}
\pagebreak
\LARGE
\underline{\textbf{\title{Tips}}}
\normalsize
\begin{table}[h]
    \centering
    \begin{tabular}{|l|l|}
        \hline
        Identidad       & \begin{tabular}[c]{@{}l@{}}$p \land V \equiv p$ \\
            $p \land F  \equiv F$            \\
            $p \lor V  \equiv V$             \\
            $p \lor F   \equiv F$\end{tabular} \\ \hline
        Idempotencia    & \begin{tabular}[c]{@{}l@{}}$p \land p \equiv p$ \\
            $p \lor p  \equiv p$\end{tabular} \\ \hline
        Involuci\'on    & \begin{tabular}[c]{@{}l@{}}$\overline{(\overline{p})}\equiv p$ \\
            $\neg{(\neg{p})} \equiv p$\end{tabular} \\ \hline
        Complemento     & \begin{tabular}[c]{@{}l@{}}$p \land \overline{p} \equiv F$ \\
            $p \lor \overline{p} \equiv V$\end{tabular} \\ \hline
        Conmutatividad  & \begin{tabular}[c]{@{}l@{}}$p \land q  \equiv q \land p$ \\
            $p \lor q \equiv q \lor p$\end{tabular} \\ \hline
        Asociatividad   & \begin{tabular}[c]{@{}l@{}}$p \land (q \land r) \equiv (p \land q) \land r$ \\
            $p \lor (q \lor r) \equiv (p \lor q) \lor r$\end{tabular} \\ \hline
        Distributividad & \begin{tabular}[c]{@{}l@{}}$p \land (q \lor r) \equiv (p \land q) \lor ( p \land r)$ \\
            $p \lor (q \land r) \equiv (p \lor q) \land ( p \lor r)$\end{tabular} \\ \hline
    \end{tabular}
    % Añadir más para la próxima semana (Morgan, transitividad y absorción)
\end{table}
\end{document}