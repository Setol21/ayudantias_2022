\documentclass[12pt]{article}
\usepackage{amsmath}
\usepackage{amsfonts}
\usepackage{amssymb}
\usepackage{vmargin}
\usepackage[utf8]{inputenc}

% \displaystyle is sooo important
\newcommand{\D}{\displaystyle}

\begin{document}
\setmargins{2.5cm}
{0.1cm}
{16.5cm}
{23.42cm}
{10pt}
{1cm}
{0pt}
{2cm}

\title{Ayudant\'ia Matem\'aticas Avanzadas I N.10}
\date{25 de Mayo 2022}
\author{Daniel S\'anchez}
\maketitle

\begin{enumerate}
	\item Determinar el dominio de
	      \begin{enumerate}
		      \item $\D h(u)=\frac{3u-5}{2u-u^2+5}$
		      \item $\D r(x)=\frac{\sqrt{2.5x-20}}{x^3+2x^2-15x}$
	      \end{enumerate}
	\item Grafique y determine el recorrido de las siguientes funciones:
	      \begin{enumerate}
		      \item $f(x)=x^2-2x+1$
		      \item $g(x)=\left\{\begin{array}{ccc}
				            x+2 & \mbox{ si } & x\leq 0 \\
				            2-x & \mbox{ si } & x>0
			            \end{array}\right.$
		      \item $h(x) = x-[x]$
	      \end{enumerate}
	\item Sea $f: A \rightarrow \mathbb{R}$ definida por $f(x) = \left\lvert \sqrt{x^2-4} - 2\right\rvert $
	      \begin{enumerate}
		      \item Determine su dominio.
		      \item ¿-1 $\in$ $Rec(f)$?
		      \item Encuentre un $x$ tal que $f(x) = 4$.
	      \end{enumerate}
	\item $g(x)=\left\{\begin{array}{ccc}
			      \sqrt{x-[x]} & \mbox{ si } & x<0 \\
			      -1           & \mbox{ si } & x=0 \\
			      1-x^2        & \mbox{ si } & x>0
		      \end{array}\right.$
\end{enumerate}
\end{document}