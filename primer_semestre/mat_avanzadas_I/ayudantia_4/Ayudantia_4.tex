\documentclass[12pt]{article}
\usepackage{amsmath}
\usepackage{amsfonts}
\usepackage{amssymb}
\usepackage{vmargin}
\usepackage[utf8]{inputenc}


\begin{document}
\setmargins{2.5cm}
{0.1cm}
{16.5cm}
{23.42cm}
{10pt}
{1cm}
{0pt}
{2cm}

\title{Ayudant\'ia Matem\'aticas Avanzadas I N.4}
\date{31 de Marzo 2022}
\author{Daniel S\'anchez}
\maketitle

\begin{enumerate}
      \item Niegue las siguientes proposiciones, dados los conjuntos $A,B \subseteq \mathbb{N}$:
            \begin{enumerate}
                  \item $(\forall x \in A)(\exists y \in B): (x-y<0 \Rightarrow x^{2}-1>9)$
                  \item $(\exists x \in A)(\forall y \in B): (x \geq 2) \Rightarrow [(xy<x) \lor (x-y \mbox{ es par})]$
                  \item $(\forall x \in B)(\forall y \in B): (x+y) \notin A \land xy \in B$
            \end{enumerate}
      \item Utilizando inducci\'on, demuestre que
            \begin{enumerate}
                  \item $\forall n \in \mathbb{N}:$
                        $$\frac{1}{1 \cdot 2}+\frac{1}{2\cdot 3}+...+\frac{1}{(n+1)(n+2)}=\frac{n+1}{n+2}$$
                  \item $\forall n \in \mathbb{N}:$
                        $$\frac{1}{2}+\frac{2}{2^2}+\frac{3}{2^3}+\frac{4}{2^4}+...+\frac{n}{2^n}=2-\frac{n+2}{2^n}$$
                  \item $\forall n \in \mathbb{N}:$
                        $$\frac{1}{1\cdot 2}+\frac{1}{2\cdot 3}+\frac{1}{3\cdot 4}+...+\frac{1}{n\cdot (n+1)}=\frac{n}{n+1}$$
                        % \item $\forall n \in \mathbb{N}: 5^{n}-1$ es un m\'ultiplo de 4.
            \end{enumerate}
\end{enumerate}


\pagebreak
\title{\LARGE{\textbf{Ejercicios repaso}}}
\maketitle
\begin{enumerate}
      \item Sea $\vartriangleright$ el conectivo definido por la siguiente equivalencia l\'ogica:
            $$(p \vartriangleright q)\equiv (p \land \neg q)\lor (\neg p \land q)$$
            Determine el valor de verdad de $p\Leftrightarrow q$ si se sabe que $p \vartriangleright (p \vartriangleright q)$ es falsa.
      \item Determine justificadamente el valor de verdad de la proposici\'on:
            $$\exists x \in A, \forall y \in B: xy \geq 0 \Rightarrow x+y>1$$
            Con $A=\{x \in \mathbb{Z}: -2<x<2\}$ y $B=\{-1,2,3,4\}$.
\end{enumerate}
\end{document}