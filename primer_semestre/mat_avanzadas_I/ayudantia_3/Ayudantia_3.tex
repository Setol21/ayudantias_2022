\documentclass[12pt]{article}
\usepackage{amsmath}
\usepackage{amsfonts}
\usepackage{amssymb}
\usepackage{vmargin}
\usepackage[utf8]{inputenc}

\begin{document}
\setmargins{2.5cm}
{0.1cm}
{16.5cm}
{23.42cm}
{10pt}
{1cm}
{0pt}
{2cm}

\title{Ayudant\'ia Matem\'aticas Avanzadas I N.3}
\date{24 de Marzo 2022}
\author{Daniel S\'anchez}
\maketitle

\begin{enumerate}
      \item Considere el conjunto referencial $E=\mathbb{N}$ y las funciones proposicionales:
            $$p(x):x\mbox{ es par, } q(x):x\geq 5$$
            Decida si las proposiciones son ambas verdaderas:
            $$\forall x \in E: p(x) \Rightarrow \neg q(x)\mbox{ ; } \exists x \in E: p(x)$$
      \item Determine el valor de verdad de las siguientes proposiciones:
            \begin{enumerate}
                  \item $(\exists x \in \{2,3,4\})(\forall y \in \{-1,0,1\})(y\geq0 \Rightarrow x+y\geq 3)$
                  \item $(\forall x \in \{1,-2,\frac{1}{2}\})(\exists y \in \{-1,1,2,3\})(xy > \frac{3}{2})$
            \end{enumerate}
      \item Considere $A=\{4,2,3,1\} \mbox{ y } B=\{-1,0,-2\}$ subconjuntos de los n\'umeros reales. Determine y justifique
            el valor de verdad de la siguiente proposici\'on:
            $$(\exists x \in A)(\forall y \in B)(xy\leq 0 \Rightarrow x^{y}<1)$$
      \item Considere los conjuntos $C=\{-1,0,1\} \mbox{ y } D=\{-2,3\}$. Determine el valor de verdad de la proposici\'on:
            $$(\exists x \in C)(\forall y \in D)(xy>0 \Rightarrow |x|<|y|)$$
\end{enumerate}
\end{document}