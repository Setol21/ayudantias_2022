\documentclass[12pt]{article}
\usepackage{amsmath}
\usepackage{amsfonts}
\usepackage{amssymb}
\usepackage{vmargin}
\usepackage[utf8]{inputenc}

% \displaystyle is sooo important
\newcommand{\D}{\displaystyle}

\begin{document}
\setmargins{2.5cm}
{0.1cm}
{16.5cm}
{23.42cm}
{10pt}
{1cm}
{0pt}
{2cm}

\title{Ayudant\'ia Matem\'aticas Avanzadas I N.13}
\date{16 de Junio 2022}
\author{Daniel S\'anchez}
\maketitle

\begin{enumerate}
	\item Considere las funciones $f: ]-\infty,1] \rightarrow \mathbb{R}$ y $g: [1,2] \rightarrow \mathbb{R}$ definidas por:
	      $$f(x) = 1 + \sqrt{1-x} \mbox{ y } g(x) = 2-x$$
	      \begin{enumerate}
		      \item Determine el recorrido de $f$.
		      \item Determine $(g \circ f)(x)$ indicando su dominio.
	      \end{enumerate}
	\item Sea $g: A \rightarrow B$ definida por $\D g(z) = \frac{z+1}{z+3}$.
	      \begin{enumerate}
		      \item Hallar dominio y recorrido de $g$.
		      \item Demuestre que $g$ es inyectiva.
		      \item ¿Es $g$ sobreyectiva? Si no lo es, redef\'inala para que lo sea y calcule $g^{-1}$.
	      \end{enumerate}
	\item Un distribuidor de art\'iculos deportivos puede adquirir camisetas de tenis a 4.000 por unidad.
	      Cuando el precio de venta se puede fijar en 10.000 se venden en promedio 6.000 unidades en un mes. El
	      distribuidor pretende subir el precio de venta y estima que por cada 1.000 de aumento en el precio, se
	      vender\'an 200 camisetas menos. 
	      
	      Considerando el precio de forma lineal:
	      \begin{enumerate}
		      \item Determine la funci\'on de las cantidades dependiendo del precio $p$.
		      \item Determine la funci\'on de utilidad dependiendo del precio $p$.
		      \item ¿A qu\'e precio se obtiene la m\'axima utilidad? Determine dicha utilidad.
	      \end{enumerate}
\end{enumerate}
\end{document}