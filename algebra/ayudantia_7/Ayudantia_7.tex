\documentclass[12pt]{article}
\usepackage{amsmath}
\usepackage{amsfonts}
\usepackage{amssymb}
\usepackage{vmargin}
\usepackage[utf8]{inputenc}

% \displaystyle is sooo important
\newcommand{\D}{\displaystyle}

\begin{document}
\setmargins{2.5cm}
{0.1cm}
{16.5cm}
{23.42cm}
{10pt}
{1cm}
{0pt}
{2cm}

\title{Ayudant\'ia \'Algebra N.6}
\date{22 de Abril 2022}
\author{Daniel S\'anchez}
\maketitle

\begin{enumerate}
      \item Demuestre las siguientes identidades:
            \begin{enumerate}
                  \item $\D \frac{1}{1-\sin \alpha} + \frac{1}{1+\sin \alpha} = 2\sec ^2 \alpha$
                  \item $(\sin \alpha + \csc \alpha)^2 + (\cos \alpha + \sec \alpha)^2 = \tan ^2 \alpha + \cot ^2 \alpha + 7$
                  \item $\cos (3\alpha) + \sin (2\alpha) - \sin (4\alpha) = (1-2\sin(\alpha))\cos(3\alpha)$
            \end{enumerate}
      \item Demuestre que si $\alpha + \beta = \dfrac{\pi}{2}$ y $\tan (\alpha) = 1-\tan (\beta)$, entonces $\sin(\alpha)\sin(\beta)=1$
      \item 
\end{enumerate}
\end{document}