\documentclass[12pt]{article}
\usepackage{amsmath}
\usepackage{amsfonts}
\usepackage{amssymb}
\usepackage{vmargin}
\usepackage[utf8]{inputenc}

% \displaystyle is sooo important
\newcommand{\D}{\displaystyle}

\begin{document}
\setmargins{2.5cm}
{0.1cm}
{16.5cm}
{23.42cm}
{10pt}
{1cm}
{0pt}
{2cm}

\title{Ayudant\'ia \'Algebra N.4}
\date{1 de Abril 2022}
\author{Daniel S\'anchez}
\maketitle

\begin{enumerate}
      \item Utilizando inducci\'on, $\forall n \in \mathbb{N}$ demuestre que:
            \begin{enumerate}
                  \item 
                        $$\D \frac{1}{1 \cdot 2}+\frac{1}{2\cdot 3}+...+\frac{1}{(n+1)(n+2)}=\frac{n+1}{n+2}$$
                  \item 
                        $$\frac{1}{2}+\frac{2}{2^2}+\frac{3}{2^3}+\frac{4}{2^4}+...+\frac{n}{2^n}=2-\frac{n+2}{2^n}$$
                  \item $5^{n}-1$ es un m\'ultiplo de 4.
                  \item $\D 11^{n+2}+12^{2n+1}$ es divisible por 133.
                  \item $2n+1\leq 3^n$
            \end{enumerate}
      \item Ejercicios propuestos:
            \begin{enumerate}
                  \item Sea $\triangleright$ el conectivo l\'ogico definido por la siguiente equivalencia l\'ogica:
                        $$(p \triangleright q)\equiv (p \land \neg q) \lor (\neg p \land q)$$
                        Determine el valor de verdad de $p \Leftrightarrow q$ si se sabe que $p \triangleright (p \triangleright q)$ es falsa.
                  \item Sea $a \in \mathbb{R}^+$ y $b \in \mathbb{R}$ tal que $b<2$. Demuestre que:
                        $$b-2a<2-ab$$
            \end{enumerate}
            % \item Exprese el k-\'esimo t\'ermino y calcule las sumas de las siguientes progresiones hasta el n-\'esimo t\'ermino:
            %       \begin{enumerate}
            %             \item $\D 5+8+11+14+17+20+...$
            %             \item $\D 1-\frac{3}{2}+\frac{9}{4}-\frac{27}{8}+...$
            %       \end{enumerate}
            % \item Calcule las siguientes sumatorias:
            %       \begin{enumerate}
            %             \item $\D \sum\limits_{k=50}^{100} (2k+4)$
            %             \item $\D \sum\limits_{k=1}^{20} \frac{1}{k(k+1)}$
            %             \item $\D \sum\limits_{k=3}^{20} \ln{\left(\frac{1}{k(k+1)}\right)}$
            %       \end{enumerate}
\end{enumerate}
\end{document}