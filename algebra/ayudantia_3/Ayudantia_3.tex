\documentclass[12pt]{article}
\usepackage{amsmath}
\usepackage{amsfonts}
\usepackage{amssymb}
\usepackage{vmargin}
\usepackage[utf8]{inputenc}

\begin{document}
\setmargins{2.5cm}
{0.1cm}
{16.5cm}
{23.42cm}
{10pt}
{1cm}
{0pt}
{2cm}

\title{Ayudant\'ia \'Algebra N.3}
\date{25 de Marzo 2022}
\author{Daniel S\'anchez}
\maketitle

\begin{enumerate}
    \item Considere $A=\{4,2,3,1\} \mbox{ y } B=\{-1,0,-2\}$ subconjuntos de los n\'umeros reales. Determine y justifique
          el valor de verdad de la siguiente proposici\'on:
          $$(\exists x \in A)(\forall y \in B)(xy\leq 0 \Rightarrow x^{y}<1)$$
    \item Considere los conjuntos $C=\{-1,0,1\} \mbox{ y } D=\{-2,3\}$. Determine el valor de verdad de la proposici\'on:
          $$(\exists x \in C)(\forall y \in D)(xy>0 \Rightarrow |x|<|y|)$$
    \item Sea $A=\{-2,0,1,2\} \mbox{ y } B=\{-1,0,1\}$. Determine el valor de verdad de la \textbf{negaci\'on}:
          $$\forall x \in A, \exists y \in B : (x+y) \notin A \land xy\in B$$
    \item Demuestre los siguientes teoremas con cualquiera de los tres m\'etodos (directo, contrarrec\'iproco o reducci\'on al absurdo):
          \begin{enumerate}
              \item $\mbox{Si } n \in \mathbb{Z}, \mbox{ si } 5n+3 \mbox{ es par}, \mbox{ entonces } n \mbox{ es impar}$.
              \item $\mbox{Si } n \in \mathbb{Z}, n^2-3 \mbox{ no es divisible por }4$.
              \item $\mbox{Si } n \in \mathbb{N}, \mbox{ entonces } n^2+n+1 \mbox{ es impar}$.
              \item $\mbox{Si }x,y \in \mathbb{R^+}, \mbox{ si } x>y \mbox{ entonces } x^2>y^2$.
          \end{enumerate}
\end{enumerate}
\end{document}