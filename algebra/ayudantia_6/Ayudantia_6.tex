\documentclass[12pt]{article}
\usepackage{amsmath}
\usepackage{amsfonts}
\usepackage{amssymb}
\usepackage{vmargin}
\usepackage[utf8]{inputenc}

% \displaystyle is sooo important
\newcommand{\D}{\displaystyle}

\begin{document}
\setmargins{2.5cm}
{0.1cm}
{16.5cm}
{23.42cm}
{10pt}
{1cm}
{0pt}
{2cm}

\title{Ayudant\'ia \'Algebra N.6}
\date{22 de Abril 2022}
\author{Daniel S\'anchez}
\maketitle

\begin{enumerate}
      \item Calcule las siguientes productorias:
            \begin{enumerate}
                  \item $\D \prod_{k=14}^{125} 2 \cdot \frac{(k^2+3k+2)}{k^2+4k+3}$
                        % Pauta prueba 2 Algebra I Comercial 2020/2
                  \item $\D \prod_{k=3}^{30} \frac{(k+2)}{k} \cdot 5^{k^2}$
                  \item $\D \prod_{k=2}^{n+1} \frac{4}{7^{k+2}}$
                  \item $\D \prod_{i=10}^{32} 3i \cdot \left(\frac{i+1}{i}\right)$
            \end{enumerate}
      \item Usando el teorema del binomio resuelva los siguientes ejercicios:
            \begin{enumerate}
                  \item Para $\D \left(\frac{3}{2}x^2-\frac{1}{3x}\right)^9$. Hallar:
                        \begin{enumerate}
                              \item El quinto t\'ermino.
                              \item El t\'ermino que contiene $x^5$.
                              \item El t\'ermino independiente $x$ (t\'ermino constante).
                        \end{enumerate}
                  \item Encuentre el t\'ermino central en el desarrollo de $\D \left({x^2 +\frac{1}{x^2}}\right)^{18}$
                  \item Determine el coeficiente de $x^6$ en el desarrollo de $\D (1+x)\left(x^2 + \frac{2}{x^3}\right)^{23}$, si es que existe.
            \end{enumerate}
\end{enumerate}


\end{document}