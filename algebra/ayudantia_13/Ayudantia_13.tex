\documentclass[12pt]{article}
\usepackage{amsmath}
\usepackage{amsfonts}
\usepackage{amssymb}
\usepackage{vmargin}
\usepackage[utf8]{inputenc}

% \displaystyle is sooo important
\newcommand{\D}{\displaystyle}

\begin{document}
\setmargins{2.5cm}
{0.1cm}
{16.5cm}
{23.42cm}
{10pt}
{1cm}
{0pt}
{2cm}

\title{Ayudant\'ia \'Algebra N.13}
\date{17 de Junio 2022}
\author{Daniel S\'anchez}
\maketitle

\begin{enumerate}
      \item Determine las soluciones de la siguiente ecuaci\'on trigonom\'etrica:
            $$2\cos^2(x)+\sin(2x)=0$$
      \item Un ni\~no est\'a haciendo volar dos cometas simult\'aneamente. Una de ellas tiene 38
            metros de cuerda y la otra 42 metros. Si el \'angulo entre las dos cuerdas es de 30°,
            estime la distancia entre los dos cometas.
      \item Considere el punto $P=(1,2,3)$ y el plano de ecuaci\'on $x+y-z=4$.
            
            Determine la ecuaci\'on de la recta que pasa por $P$ y es perpendicular al plano dado.
            
            ¿En qu\'e punto corta esta recta al plano?
      \item Determine la ecuaci\'on general del plano que contiene al punto de coordenadas $(0,1,1)$
            y la recta $L$ de ecuaci\'on:
            $$L=\left\{ \begin{array}{cccl}
                        x & = & -t                                    \\
                        y & = & 3+2t & \mbox{ con $t \in \mathbb{R}$} \\
                        z & = & -5-t
                  \end{array}\right.$$
\end{enumerate}
\end{document}