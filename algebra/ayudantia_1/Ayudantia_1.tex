\documentclass[12pt]{article}
\usepackage{amsmath}
\usepackage{amsfonts}
\usepackage{vmargin}
\usepackage[utf8]{inputenc}

\setmargins{2.5cm}
{0.1cm}
{16.5cm}
{23.42cm}
{10pt}
{1cm}
{0pt}
{2cm} 

\begin{document}
\title{Ayudant\'ia \'Algebra N.1}
\date{10 de Marzo 2022}
\author{Daniel S\'anchez}
\maketitle

\begin{enumerate}
    \item Determine si las siguientes frases son una proposici\'on, de ser
          as\'i identifique su valor de verdad:
          \begin{enumerate}
              \item Si $6<3$ entonces $30>10$.
              \item Si el cuadrado de $7$ es un n\'umero par entonces $7$ es par.
              \item $5$ es distinto de $6$ y $24$ es mayor que $36$.
              \item ¿Qu\'e d\'ia es hoy?
          \end{enumerate}
    \item Determine si las siguientes expresiones son tautolog\'ias, contradicci\'on
          o contingencia:
          \begin{enumerate}
              \item $(p \land (p \Rightarrow q)) \Rightarrow q$
              \item $[(p \Rightarrow q) \land p \land \lnot q] \Rightarrow (\lnot p \lor q)$
          \end{enumerate}
    \item Determine el valor de verdad de $p, q, r\mbox{ y } s$, si la proposici\'on:
          \begin{enumerate}
              \item $[(r \Rightarrow (s \land p))]\Rightarrow [(q \lor r) \Rightarrow q]$ es falsa.
              \item Con los valores de verdad de $p, q, r \mbox{ y } s$, resuelva:
                    \begin{enumerate}
                        \item $p \land (r \lor q)$
                        \item $(r \Rightarrow \lnot q) \Rightarrow \lnot q$
                    \end{enumerate}
          \end{enumerate}
    \item Sea $p \equiv F \mbox{ ; } q \equiv V \mbox{ y } r \equiv F$, determine el valor de verdad de la siguiente proposici\'on:
          \begin{enumerate}
              \item $[(p \lor q) \land (p \Rightarrow r)] \Rightarrow [(p \land q)\lor(q \Rightarrow r)]$
              \item $[p \land (p \Rightarrow r)] \Rightarrow [(\bar{r} \lor q)\land \bar{r}]$
          \end{enumerate}
\end{enumerate}
\pagebreak
\LARGE
\underline{\textbf{\title{Tips}}}
\normalsize
\begin{table}[h]
    \centering
    \begin{tabular}{|l|l|}
        \hline
        Identidad       & \begin{tabular}[c]{@{}l@{}}$p \land V \equiv p$ \\
                              $p \land F  \equiv F$            \\
                              $p \lor V  \equiv V$             \\
                              $p \lor F   \equiv F$\end{tabular}                                                                \\ \hline
        Idempotencia    & \begin{tabular}[c]{@{}l@{}}$p \land p \equiv p$ \\
                              $p \lor p  \equiv p$\end{tabular}                                                                \\ \hline
        Involuci\'on    & \begin{tabular}[c]{@{}l@{}}$\overline{(\overline{p})}\equiv p$ \\
                              $\neg{(\neg{p})} \equiv p$\end{tabular}                                                    \\ \hline
        Complemento     & \begin{tabular}[c]{@{}l@{}}$p \land \overline{p} \equiv F$ \\
                              $p \lor \overline{p} \equiv V$\end{tabular}                                           \\ \hline
        Conmutatividad  & \begin{tabular}[c]{@{}l@{}}$p \land q  \equiv q \land p$ \\
                              $p \lor q \equiv q \lor p$\end{tabular}                                                       \\ \hline
        Asociatividad   & \begin{tabular}[c]{@{}l@{}}$p \land (q \land r) \equiv (p \land q) \land r$ \\
                              $p \lor (q \lor r) \equiv (p \lor q) \lor r$\end{tabular}                           \\ \hline
        Distributividad & \begin{tabular}[c]{@{}l@{}}$p \land (q \lor r) \equiv (p \land q) \lor ( p \land r)$ \\
                              $p \lor (q \land r) \equiv (p \lor q) \land ( p \lor r)$\end{tabular}            \\ \hline
        Leyes de Morgan & \begin{tabular}[c]{@{}l@{}}$\overline{(p \lor q)} \equiv \overline{p} \land \overline{q}$ \\
                              $\overline{(p \land q)} \equiv \overline{p} \lor \overline{q}$\end{tabular} \\ \hline
        Transitividad   & \begin{tabular}[c]{@{}l@{}}$[(p \Rightarrow q) \land (q \Rightarrow r)] \equiv (p \Rightarrow r)$\end{tabular}        \\ \hline
        Absorci\'on     & \begin{tabular}[c]{@{}l@{}}$[p \land (p \lor q)]\equiv p$ \\
                              $[p \lor (p \land q)]\equiv p$\end{tabular}                                    \\ \hline
    \end{tabular}
\end{table}
\end{document}