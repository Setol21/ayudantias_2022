\documentclass[12pt]{article}
\usepackage{amsmath}
\usepackage{amsfonts}
\usepackage{amssymb}
\usepackage{vmargin}
\usepackage[utf8]{inputenc}

% \displaystyle is sooo important
\newcommand{\D}{\displaystyle}

\begin{document}
\setmargins{2.5cm}
{0.1cm}
{16.5cm}
{23.42cm}
{10pt}
{1cm}
{0pt}
{2cm}

\title{Ayudant\'ia Matem\'aticas Avanzadas I N.11}
\date{2 de Junio 2022}
\author{Daniel S\'anchez}
\maketitle

\begin{enumerate}
	\item Considere las funciones
	      $f: [4,\infty[\rightarrow \mathbb{R}$ y 
	      $g: [-2,\infty[\rightarrow \mathbb{R}$ dadas por 
	      $$f(x)=x^2-2x+3 \mbox{ y } g(x)=\sqrt{x+2}+1$$
	      \begin{enumerate}
		      \item Determine si las funciones $f \mbox{ y } g$
		            son inyectivas, sobreyectivas y/o biyectivas.
		      \item Calcule $(f \circ g)(x)$ indicando claramente su dominio.
	      \end{enumerate}
	\item Considere la funci\'on
	      $f: ]1,\infty[ \rightarrow ]0,\infty[$ dada por
	      $\D f(x)=\frac{1}{\sqrt{x-1}}$
	      \begin{enumerate}
		      \item Demuestre que $f$ es biyectiva.
		      \item Determine la funci\'on inversa de
		            $f$ con su dominio y recorrido.
		      \item Grafique $f$ y su inversa.
	      \end{enumerate}
\end{enumerate}
\end{document}