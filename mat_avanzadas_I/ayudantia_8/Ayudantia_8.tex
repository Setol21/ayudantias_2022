\documentclass[12pt]{article}
\usepackage{amsmath}
\usepackage{amsfonts}
\usepackage{amssymb}
\usepackage{vmargin}
\usepackage[utf8]{inputenc}

% \displaystyle is sooo important
\newcommand{\D}{\displaystyle}

\begin{document}
\setmargins{2.5cm}
{0.1cm}
{16.5cm}
{23.42cm}
{10pt}
{1cm}
{0pt}
{2cm}

\title{Ayudant\'ia Matem\'aticas Avanzadas I N.8}
\date{28 de Abril 2022}
\author{Daniel S\'anchez}
\maketitle

\begin{enumerate}
      \item Sumatorias
            \begin{enumerate}
                  \item Calcule el valor de $\D \sum_{k=15}^{100}\left[\dfrac{1}{k^2+k}+\dfrac{2^{k+2}}{3^k}\right]$
                  \item Calcule el valor de $\D \sum_{k=1}^{n}2\sqrt{3}\left[1-\dfrac{1}{2^k}\right]$
                  \item Determine el valor de $n\in \mathbb{N}$ para que $\D \sum_{k=1}^{n} (2k+8)=136$
            \end{enumerate}
      \item Progresiones
            \begin{enumerate}
                  \item Sea $(a_n)_{n\in\mathbb{N}}$ una sucesión real tal que $a_1 = 1$ y para todo $n>1$ se cumple que $a_{n+1} = (-1)^{n+1} + a_n$. Calcule:
                        \begin{itemize}
                              \item $\D \sum_{i=1}^{100}a_i$
                              \item $\D \sum_{i=1}^{100}a_{2i-1}$
                        \end{itemize}
                  \item Dada la sucesión de números definida por: $a_1 = 3; a_{n+1}=a_n+3(n+1), n \in\mathbb{N}$. Calcule el valor de la siguiente sumatoria: $\D \sum_{i=1}^{150}a_i$
                  \item Si se sabe que: $\D \sum_{k=1}^{24}{2a_k}^2=560, \D \sum_{k=1}^{24}(3a_k-2)= 125, a_{25} = -4$ y que $a_{26} = 7$. Calcule el valor de la siguiente sumatoria: $\D \sum_{k=1}^{26}(2a_k-3)^2$
            \end{enumerate}
            \pagebreak
      \item Números Reales
            \begin{enumerate}
                  \item Sean $a,b \in\mathbb{R}^{+}$ tales que $a>b$. Determine usando los axiomas y propiedades de orden de los números reales, si es verdadera o falsa la desigualdad: $$\frac{a^{3}b-ab^{3}}{b-a}>0.$$
                  \item Sean $a,b,c,d\in\mathbb{R}$ tales que: $$a-2b>0 \wedge 3c-d>0$$ Determine usando los axiomas de orden de los números reales, si es verdadera o falsa la desigualdad: $$3ac+2bd>ad+6cb.$$
                  \item $\forall a,b \in\mathbb{R}^{+}$, demuestre:
                        \begin{itemize}
                              \item $(a+b)(a^{-1}+b^{-1})\geq4$.
                              \item $a+a^{-1}\geq1$.
                        \end{itemize}
            \end{enumerate}
      \item Inecuaciones y desigualdades
            \begin{enumerate}
                  %PAUTA 2017 CONTROL 1 C. DIFERENCIAL
                  \item $\dfrac{1}{2x-5}\geq1$
                  \item $\dfrac{x}{x-1}\geq \dfrac{x-1}{x}$ %control 1 calculo 2020-1
                  \item $\dfrac{x^{2}+4}{x^{2}-4}\geq0$
            \end{enumerate}
      \item Valor absoluto
            \begin{enumerate}
                  \item $|{2x-1}|<3$
                  \item $|{5-|{x-2}|}|<7$
                  \item $\left|{\dfrac{x}{1-x}}\right|\leq\dfrac{1}{x}$
            \end{enumerate}
\end{enumerate}
\end{document}