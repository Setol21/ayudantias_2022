\documentclass[12pt]{article}
\usepackage{amsmath}
\usepackage{amsfonts}
\usepackage{amssymb}
\usepackage{vmargin}
\usepackage[utf8]{inputenc}

% \displaystyle is sooo important
\newcommand{\D}{\displaystyle}

\begin{document}
\setmargins{2.5cm}
{0.1cm}
{16.5cm}
{23.42cm}
{10pt}
{1cm}
{0pt}
{2cm}

\title{Ayudant\'ia Matem\'aticas Avanzadas I N.12}
\date{9 de Junio 2022}
\author{Daniel S\'anchez}
\maketitle

\begin{enumerate}
	\item Considere $\D f: A \rightarrow B$ dada por $\D f(x) = \ln\left(\frac{x}{x-4}\right)$
	      \begin{enumerate}
		      \item Determine el dominio y recorrido de $f$.
		      \item Demuestre que $f$ es biyectiva y determine su funci\'on inversa.
	      \end{enumerate}
	\item Dada la funci\'on $f: \mathbb{R} \rightarrow Rec(f)$, donde:
	      $$f(x) = \left\{\begin{array}{ccc}
			      x+1   & \mbox{ si } & x\leq 0 \\
			      x^2+b & \mbox{ si } & x > 0
		      \end{array}\right.$$
	      \begin{enumerate}
		      \item ¿Para qu\'e valor(es) de $b$ es $f$ inyectiva? Justifique utilizando la gr\'afica.
		      \item Para que el valor $b=2$, encuentre $f^{-1}$ indicando su dominio.
	      \end{enumerate}
	\item Si el precio $p$ que se le fija a un art\'iculo depende de la cantidad $q$ demandada, verificando
	      la siguiente ley de demanda:
	      $$p = 120 - \frac{q}{5}$$
	      \begin{enumerate}
		      \item Determine la funci\'on del ingreso por la venta de este art\'iculo, dependiendo del precio $p$.
		      \item Determine el precio que debe fijarse de tal forma que el ingreso sea el m\'aximo posible.
	      \end{enumerate}
\end{enumerate}
\end{document}